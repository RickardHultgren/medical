\documentclass[13pt,a4paper,oneside]{article}
\usepackage[a4paper,portrait,left=2.1cm,right=2.1cm,top=1.8cm,bottom=1.8cm]{geometry}
\usepackage{graphicx}
\usepackage[swedish]{babel}
%\DeclareMathAlphabet{\mathpzc}{OT1}{pzc}{m}{it}
%\pretocmd{\scshape}{\fontfamily{pplx}\selectfont}{}{}
\usepackage{mathpazo}
%\usepackage[latin1]{inputenc}
%\usepackage{iwona}
%\usepackage[math]{iwona}
\usepackage[T1]{fontenc}
\usepackage[utf8]{inputenc}
\usepackage{microtype}
\renewcommand{\oldstylenums}[1]{{\fontfamily{pplj}\selectfont #1}}
\newcommand\sho[1]{{\tiny\tt\char92#1}}
\renewcommand{\oldstylenums}[1]{{\fontfamily{pplj}\selectfont #1}}
\usepackage{multicol}
\usepackage[T1]{fontenc}
\usepackage{titlesec}
%\usepackage{\sectsty}
%\allsectionsfont{\sffamily}
\usepackage{enumitem}
\setlist{leftmargin=2cm}
\DeclareMathAlphabet{\mathpzc}{OT1}{pzc}{m}{it}
\usepackage{hyperref}
\usepackage{url}
\setlength{\columnsep}{1.5cm}
\setlength{\columnseprule}{0.2pt}
\usepackage{titlesec}
\usepackage{mathpazo}
\usepackage[T1]{fontenc}
\usepackage{tikz}
\usetikzlibrary{shapes}
\usepackage[math]{iwona}
\usepackage{enumitem,amssymb}
\newlist{todolist}{itemize}{2}
\setlist[todolist]{label=$\square$}
\newcommand{\mypicture}[1][]{% 1 optional parameter for options for the tikz picture
\begin{tikzpicture}[#1]
\node[regular polygon, regular polygon sides=8,draw=red, double, double distance=0.5mm, ultra thick,fill=red, font=\tiny\bfseries, text=white,inner sep=0mm]{STOP}
\end{tikzpicture}
} 

%\titleformat*{\section{\bfseries}}
\titleformat*{\subsection}{\bf\sffamily\large}
%\newcommand\sho[1]{{\tiny\tt\char92#1}}
\pagestyle{empty}\begin{document}%\centering
{
\ \\
Inskrivning ortopeden\hrule
\ \\
\begin{tabular}{ l l }
 \begin{minipage}{12cm} \begin{itemize}
  \item Tryck \textit{Ctrl + B} för att få \textit{Besöklistan}. Välj sedan följande:
\begin{itemize}
  \item Vårdande enhet: \textbf{Inskrivningscentrum ortopedi Arvika}
  \item Vårdpersonal: \textbf{Alla}
  \item Tryck \textbf{uppdatera}
\end{itemize}\end{itemize} \end{minipage} &  \begin{minipage}{12cm}\end{minipage} \\ 
  \begin{minipage}{12cm}\begin{itemize}

  \item Fyll i följande uppgifter från \textit{Besökslistan}:
\begin{itemize}
\item Namn: \\\ \\\ ...............................................................................................................
  \item Område
    \begin{todolist}  
        \item Rygg
        \item Knä
        \item Höft
    \end{todolist}
    \item Sida:
    \begin{todolist}  
        \item Höger
        \item Vänster
        \item Bilateralt
        \end{todolist}
    \end{itemize}
  \item [\mypicture] {\sc Kontraindicerande;} Tryck \textit{Ctrl + \sc{7}} för att få \textit{Patientöversikt}. Undersök följande:
    \begin{todolist}  
        \item {\sc Bmi} $>$ {\sc 35}?
        \item Patologier som kontraindicerar operation?
        \begin{todolist}  
            \item Möjliga åtgärder?
        \end{todolist}
        \item Mediciner som kontraindicerar operation?
        \begin{todolist}  
            \item Möjliga åtgärder?        
        \end{todolist}
    \end{todolist}
    \item Blodtryck: \\\ \\\  \\\ ..................................../.................................... mmHg
    \item Puls \\\  \\\ \\\ ....................................{\sc bpm}
    \item S-Zink\\\ \\\ \\\ .................................... $\leq$ 9 $\rightarrow$ Solvezink 
    \item[$\square$] {\sc Ekg} sinusrytm.
    \item[$\square$] {\sc Ekg} Patologier:
        \begin{todolist} 
            \item Skänkelblock
                \begin{todolist} 
                    \item Högersidigt
                    \item Vänstersidig              
            \end{todolist}         
            \item
            \item Annat \\\  \\\ \\\ ................................................................................................................................................ 
        \end{todolist}         
    \item Skriv ut läkemedelslista.
\end{itemize}\end{minipage} &  \begin{minipage}{12cm}\end{minipage}  
  %\begin{minipage}{12cm}\end{minipage} &  \begin{minipage}{12cm}\end{minipage}    
\end{tabular}
%\subsection*{Rubrik}

\newpage

\\
\end{document}
