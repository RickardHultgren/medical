\documentclass[13pt,a4paper,oneside]{article}
\usepackage[a4paper,portrait,left=2.1cm,right=2.1cm,top=1.8cm,bottom=1.8cm]{geometry}
\usepackage{graphicx}
\usepackage[swedish]{babel}
%\DeclareMathAlphabet{\mathpzc}{OT1}{pzc}{m}{it}
%\pretocmd{\scshape}{\fontfamily{pplx}\selectfont}{}{}
\usepackage{mathpazo}
%\usepackage[latin1]{inputenc}
%\usepackage{iwona}
%\usepackage[math]{iwona}
\usepackage[T1]{fontenc}
\usepackage[utf8]{inputenc}
\usepackage{microtype}
\renewcommand{\oldstylenums}[1]{{\fontfamily{pplj}\selectfont #1}}
\newcommand\sho[1]{{\tiny\tt\char92#1}}
\renewcommand{\oldstylenums}[1]{{\fontfamily{pplj}\selectfont #1}}
\usepackage{multicol}
\usepackage{tabularx}
\usepackage[T1]{fontenc}
\usepackage{titlesec}
%\usepackage{\sectsty}
%\allsectionsfont{\sffamily}
\usepackage{enumitem}
\setlist{leftmargin=2cm}
\DeclareMathAlphabet{\mathpzc}{OT1}{pzc}{m}{it}
\usepackage{hyperref}
\usepackage{url}
\setlength{\columnsep}{1.5cm}
\setlength{\columnseprule}{0.2pt}
\usepackage{titlesec}
\usepackage{mathpazo}
\usepackage[T1]{fontenc}
\usepackage{tikz}
\usetikzlibrary{shapes}
\usepackage[math]{iwona}
\newsavebox{\X}
\sbox{\X}{\textbf{\colorbox{yellow}{}}}
\usebox{\X}
\def\comment{\colorbox{yellow}}{}
\usepackage{enumitem,amssymb}

\newlist{todolist}{itemize}{2}
\setlist[todolist]{label=$\square$}

\newlist{htodolist}{itemize*}{2}
\setlist[htodolist]{label=\ $\ \square$}


\newcommand{\mypicture}[1][]{% 1 optional parameter for options for the tikz picture
\begin{tikzpicture}[#1]
\node[regular polygon, regular polygon sides=8,draw=red, double, double distance=0.5mm, ultra thick,fill=red, font=\tiny\bfseries, text=white,inner sep=0mm]{STOP}
\end{tikzpicture}
} 

%\titleformat*{\section{\bfseries}}
\titleformat*{\subsection}{\bf\sffamily\large}
%\newcommand\sho[1]{{\tiny\tt\char92#1}}
\pagestyle{empty}\begin{document}%\centering
{
\ \\
Inskrivning ortopeden\hrule
\ \\
\ \\
\begin{tabularx}{\textwidth}{p{15em} p{1.5cm}|X}
\hline \begin{minipage}{5cm}Tryck \textit{Ctrl + B} för att få \textit{Besöklistan}. Välj sedan följande:
\begin{itemize}[leftmargin=1.5em]
  \setlength\itemsep{.25em}
  \item Vårdande enhet: \textbf{Inskrivningscentrum ortopedi Arvika}
  \item Vårdpersonal: \textbf{Alla}
  \item Tryck \textbf{uppdatera}
\end{itemize} \end{minipage} & \ & \ \\
\hline\multicolumn{3}{c}{\begin{minipage}{30cm}
\comment{{\bf Kontaktorsak:} Inskrivningssamtal inför planerad ...}\\
Fyll i följande uppgifter från \textit{Besökslistan}:
\begin{itemize}
\item Namn: \\ \vspace{0cm}\hspace{1cm} ...................................................................................................................................................
  \item Område:
    \begin{htodolist}  
        \item Rygg 
        \item Knä 
        \item Höft 
    \end{htodolist}
    \item Sida:
    \begin{htodolist}  
        \item Höger
        \item Vänster
        \item Bilateralt
        \end{htodolist}
\end{itemize}
Kolla Provisio:
\begin{itemize}
        \item Operations-datum: \\ \vspace{3em}\hspace{3cm} ...............................................................................................................................\vspace{-2em}
\end{itemize}\end{minipage}}\\
\hline & \ & \begin{minipage}{10cm} \comment{{\bf Social situation:}     \begin{htodolist}  
        \item Ensamboende
        \item Hemhjälp
        \item Trappor
    \end{htodolist}}\end{minipage}\\\
 \ & \ & \begin{minipage}{10cm}\comment{{\bf Levnadsvanor:} Ej rökning / alkohol 6 v innan -› 6 v efter}\end{minipage}\\
 \hline\multicolumn{3}{l}{\comment{\bf Tid. och nuv. sjd:}\vspace{3em}}\\
 \hline \ & \ & \begin{minipage}{10cm}\comment{{\bf Uppmärksamhetssignal:}}\\\ 
 \comment{
 \begin{htodolist}  
        \item Antibiotika
        \item Lokalbedövning
        \item Nickel
        \item {\sc Nsaid}
    \end{htodolist}}\end{minipage}\\
\hline\multicolumn{3}{l}{\comment{{\bf Läkemedelslista:} Aktuell lista}\vspace{3em}}\\
 \hline\ & \ & \begin{minipage}{10cm}\comment{{\bf Aktuellt:}
 \begin{htodolist}  
        \item Smärta belastning
        \item Smärta vila
        \item Smärta nattetid
\end{htodolist}}
 \comment{\begin{htodolist}
        \item Hjälpmedel
        \item Tidigare kortison-injektion
    \end{htodolist}}
    \end{minipage}\\
\hline\multicolumn{3}{c}{\begin{minipage}{30cm}
 Tid. kortisoninj?\\
\begin{minipage}{10cm}  Blodtryck: \\ \vspace{0cm}\hspace{1.5cm} .............................../.......................... mmHg
    \end{minipage}
\begin{minipage}{10cm} 
\item Puls: \\ \vspace{0cm}\hspace{1.5cm} ....................................{\sc bpm}
    \end{minipage}
    \begin{itemize}
    \item[$\square$] \comment{Allmäntillstånd}
    \item[$\square$] \comment{Hjärta. Normalfrekvent, regelbunden puls. Inga bi- eller blåsljud}
    \item[$\square$] \comment{Lungor. Normalfrekvent andning. Vescikulära andningljud bilateralt.}    
    \item[$\square$] \comment{Hud}    
    \item[$\square$] \comment{\sc Ekg} Sinusrytm
    \item[$\square$] \comment{\sc Ekg} Patologier:
        \begin{todolist} 
            \item Skänkelblock
                \begin{htodolist} 
                    \item Högersidigt
                    \item Vänstersidig              
            \end{htodolist}         
            \item Annat \\ \vspace{0cm}\hspace{1.5cm} ......................................
            
        \end{todolist}         
    \item[$\square$] S-Zink \\ \vspace{0cm}\hspace{1.5cm} ............................... $<$ 11 $\rightarrow$ Solvezink tabl 1x2 i3v från op (ej tidssatt)
    \item[$\square$] Högt CRP omkontroll
    \item[$\square$] Kalium kontroll op-dag om avvikande
\end{itemize}\end{minipage} }
\end{tabularx}\newpage\\
\begin{tabularx}{\textwidth}{p{15em} p{1.5cm}|X}
\ & \ & \begin{minipage}{10cm}\comment{Bedömning: I samråd med narkosen bedöms pat som kandidat för operation}\end{minipage}\\
 \ & \ & \begin{minipage}{10cm}{comment{}Läkemedelsförändring: Enligt mall för höft/knäplast}\end{minipage}\\
 \ & \ & \begin{minipage}{10cm}Premedicinering med ... ... på op-dage\end{minipage}\\
 \ & \ & \begin{minipage}{10cm}
Kryssa läkemedel på op-dagen 
- beta-bl och inhalationer kryssas i regel inte
- Trombyl kryssas endast op-dag
- Clopidogrel ut 7 dagar innan op och in när Eliquis utsättes 
- ej NSAID 1 v innan

Paket: Ej tidssitt!
Arcoxia om ej kontraindicerat (magsår, allergi ... ?)
Gabapentin fel dos - ändra Eliquis förstahandsval
(Fragmin ev om GFR<30)
Remiss till AK-mottagningen (medicinmott Arvika ...) om pat har NOAK el Waran (välj inget i paketet)
\end{minipage}\\
 \ & \ & \begin{minipage}{10cm}Diagnoskod:
M161 höft 
M171 knä
\end{minipage}

\end{tabularx}


\end{document}
